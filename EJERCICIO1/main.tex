%! TEX root = ./main.tex
\documentclass{article}

\usepackage{graphicx} % Required for inserting images
\graphicspath{ {./images/} }
\usepackage{amsmath, amssymb, amsthm}
\usepackage{mathtools}
\usepackage[english]{babel}
\setlength{\parindent}{0pt}
\newtheorem*{prop}{Proposición}
\newcommand{\bigslant}[2]{{\raisebox{.2em}{$#1$}\left/\raisebox{-.2em}{$#2$}\right.}}

\title{Ejercicio 1}

\author{JOAQUIN ANDRE PUENTE GRAJEDA}
\date{April 2025}

\begin{document}

    \maketitle

    \section*{Ejercicio 1}
    \subsection*{a) Propuesta}
    Para la ejecución del algoritmo se propone la ejecución del programa que
    tome menor tiempo de primero, seguido de los programas de menor tiempo. Es decir,
    la ejecución de los programas en órden de tiempo de ejecución ascendente.

    \subsection*{b) Tiempo de ejecución}
    La tarea se divide en dos pasos, el primer paso es ordenar el conjunto de tareas,
    y el segundo paso es ejecutar cada una de las tareas.\\
    Utilizando Merge Sort podemos ordenar el conjunto de tareas con un tiempo $O(n \cdot logn)$\\
    El segundo paso es calendarizar y ejecutar las tareas el cual toma un tiempo de $O(n)$\\
    $\therefore$ El tiempo de ejecución del algoritmo es $O(n \cdot logn)$
    
    \subsection*{c) Demostración es óptimo}
    \begin{proof}
        Supongamos que tenemos un calendario donde tenemos dos tareas $a_i$, $a_j$ tales
        que su tiempo de ejecución $p_i > p_j$. Los tiempos de ejecución de ejecución son:
        \[
            c_i = t + p_i
        \]
        \[
            c_j = t + p_i + p_j 
        \]
        Al sumar los tiempos tenemos $2t+2p_i+p_j$
        Si cambiamos el órden de ejecución de $a_i$ y $a_j$ tendríamos el tiempo
        total de $2t+2p_j+p_i$. Y dado que $p_i > p_j$, el tiempo de ejecución es
        menor en el primer caso. Por lo tanto el ordenamiento de las tareas en tiempo
        de ejecución si afacta y logra la mínima suma posible.


    \end{proof}

\end{document}
